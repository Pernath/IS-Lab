%Especificacion
\documentclass[12pt]{article}

%Paquetes
\usepackage[left=2cm,right=2cm,top=3cm,bottom=3cm,letterpaper]{geometry}
\usepackage{lmodern}
\usepackage[T1]{fontenc}
\usepackage[utf8]{inputenc}
\usepackage[spanish,activeacute]{babel}
\usepackage{mathtools}
\usepackage{amssymb}
\usepackage{enumerate}

%Preambulo
\title{Ingeniería de Software \\ Laboratorio: Práctica 1}
\author{Carlos Gerardo Acosta Hernández}
\date{Entrega: 08/03/16 \\ Facultad de Ciencias UNAM}

\begin{document}
\maketitle
\section{Cuestionario}
\begin{enumerate}
\item \textbf{¿Cuánto tiempo te llevó resolver la práctica} \par
  Alrededor de dos horas en la implementación, pero la investigación previa para poder realizar efectivamente el mapeo lo hice ocasionalmente estas dos semanas que pasaron antes de la entrega.
\item \textbf{¿Cómo se llaman los archivos de configuraciones de Hibernate, tuviste alguna dificultad con la configuración de estos? } \par
  El archivo de configuración general de $Hibernate$: \textit{hibernate.cfg.xml} fue autogenerado. Ubicado en \textit{src/java}.\par
  Para la tabla $Usuario$, el archivo correspondiente es ``MapeoUsuario.hbm.xml'', para la tabla $Telefono$, el archivo es ``MapeoTelefono.hbm.xml'' y el mapeo para la tabla $Tipos$ está en ``MapeoTipos.hbm.xml''. Todos estos ubicados en \textit{src/java/Modelo}. \par
  En realidad, con el \textit{Hibernate Mapping Wizard}, me sentí bastante guíado, sólo hay un par de cosas que no me quedan claras si se deben reflejar directamente en el xml de mapeo de clase, como las restricciones.  \\
\end{enumerate}
\end{document}
